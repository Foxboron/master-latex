% !TEX encoding = UTF-8 Unicode
%!TEX root = thesis.tex
% !TEX spellcheck = en-US
%%=========================================
\documentclass[../Main/thesis.tex]{subfiles}
\begin{document}
\chapter{APT Testing setup}%
\label{appendix:apt_testing}
To run the APT transport test suite you need to have a running docker daemon and
the correct privileges set up. Below is the expected abbreviated output of the
test run.

\begin{minted}{bash}
$ git clone https://github.com/Foxboron/master.git
Cloning into 'master'...
$ cd master 
$ make transport-test                                
[*] Starting
[*] Waiting for system to start...
[*] Adding test data...
[*] Running test suite...
[...]
apt_1       | py37 create: /app/.tox/py37
apt_1       | py37 installdeps: -rrequirements.txt, pylint, bandit, coverage, mock
apt_1       | py37 installed: asn1crypto==0.24.0,astroid==2.2.5,attrs==19.1.0,bandit==1.6.0,certifi==2019.3.9,cffi==1.12.3,chardet==3.0.4,colorama==0.4.1,coverage==4.5.3,cryptography==2.6.1,gitdb2==2.0.5,GitPython==2.1.11,idna==2.8,in-toto==0.3.0,iso8601==0.1.12,isort==4.3.20,lazy-object-proxy==1.4.1,mccabe==0.6.1,mock==3.0.5,pathspec==0.5.9,pbr==5.2.0,pycparser==2.19,pylint==2.3.1,PyNaCl==1.3.0,python-dateutil==2.8.0,PyYAML==5.1,requests==2.22.0,securesystemslib==0.11.3,six==1.12.0,smmap2==2.0.5,stevedore==1.30.1,typed-ast==1.3.5,urllib3==1.25.2,wrapt==1.11.1
apt_1       | py37 run-test: commands[2] | coverage run -m unittest discover
[...]
apt_1       | ----------------------------------------------------------------------
apt_1       | Ran 11 tests in 18.749s
apt_1       | OK
[...]
apt_1       |   py37: commands succeeded
apt_1       |   congratulations :)
[...]
[*] Removing containers...
\end{minted}
\end{document}
