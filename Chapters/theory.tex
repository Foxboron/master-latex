% !TEX encoding = UTF-8 Unicode
%!TEX root = ../Main/thesis.tex
% !TEX spellcheck = en-US
%%=========================================
\documentclass[../Main/thesis.tex]{subfiles}
\begin{document}
\chapter{Theory}
\label{ch:theory}
In this section we will take a closer look at the theory surrounding
supply chains, reproducible builds, ...

\section{Supply Chains}\label{sec:supply_chain}
    \subsection*{The Supply Chain}
    The supply chain is a system of delivering products to users. In the real of
    software development the product is the given software you are distributing.
    The supply chain in this regard is the complete steps from developers
    writing code. The code is committed to a repository or central storage. This
    code is then tested and prepared for release. The release can either be by
    making the product available for download, or by deploying the software to
    the customers.

    TODO: Find a proper definition of supply chain to add

    \subsubsection*{in-toto}
    in-toto is a framework to verify the integrity of a supply chain. It defines
    a specification that details what steps should occur. As one supply chain
    could define and utilize any number of steps its vital for this to be
    extensible, and customize able. In-toto lets the specification detail who
    should perform the step in the supply chain.
  
    The layout describes what each step of the supply chain, along with a set of
    signatures and keys that defines the valid authorizes of the given step.

    TODO: Add an example of a layout

    The link metadata is a JSON file that specifies what the values, and outputs
    of the corresponding step should be. Evaluating the specification along with
    the link metadata lets the users, or the organization, verify that the
    supply chain has not been tampered with. 

    TODO: Add an example of link-metadata

\section{Linux Distributions}\label{sec:linux_distributions}
    \subsection*{Linux}
    Linux is free and open-source project that implements a kernel. It was first
    developed by Linus Torvalds in the early 1990 and has grown into the largest
    open-source project today. It is commonly used in everything from firmware
    modules on a computer, to the every increasing field of Internet of Thing,
    along with servers and on personal computers.

    Linux is accompanied by a suit of tools and environment that is commonly
    referred to as a ''Distribution'' and defines an operating system based on
    Linux. These are created by companies as commercial products, as well as
    groups of volunteers as a hobby for free. A lot of open-source development
    is done with, and for, Linux.

    \subsection*{Debian}
    Debian was one of the first operating systems based on Linux, and was
    created by Ian Murdock in 1993.

    \subsection*{apt and package manager}
    One of the main innovations from debian was the creation of the very first
    package manager. Package manager allows users to download pre-compiled
    software from centralized repositories maintained by the Debian developers.
    This allows users to easily fetch, update and remove installed packages on
    their system.


\section{Reproducible Builds}\label{sec:reproducible_builds}
    \subsection*{Determinism}
    \subsection*{Source Date Epoch}
    \subsection*{Buildinfo}

\section{Rebuilders}\label{sec:rebuilders} 

\section{Merkle Trees}
Merkle Trees are....

    \subsection*{Blockchain}

    \subsubsection*{Certificate transparency log}


\section{Development}\label{sec:development} 
    \subsection*{Python}
    \subsection*{sqlalchemy}
    \subsection*{Panda and matplotlib}

\section{Summary}\label{sec:summary-theory} 

\blankpage
\end{document}
