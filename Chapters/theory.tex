% !TEX encoding = UTF-8 Unicode
%!TEX root = ../Main/thesis.tex
% !TEX spellcheck = en-US
%%=========================================
\documentclass[../Main/thesis.tex]{subfiles}
\begin{document}
\chapter{Theory}
\label{ch:theory}
In this section we will take a closer look at the theory surrounding
supply chains, reproducible builds, transparency logs and rebuilders.

\section{Supply Chains}\label{sec:supply_chain}
    \subsection*{The Supply Chain}
    The supply chain is a system of delivering products to users. In the real of
    software development the product is the given software you are distributing.
    The supply chain in this regard is the complete steps from developers
    writing code. The code is committed to a repository or central storage. This
    code is then tested and prepared for release. The release can either be by
    making the product available for download, or by deploying the software to
    the customers.

    TODO: Find a proper definition of supply chain

    \subsubsection*{in-toto}
    in-toto is a framework to verify the integrity of a supply chain. It defines
    a specification that details what steps should occur. As one supply chain
    could define and utilize any number of steps its vital for this to be
    extensible, and customize able. In-toto lets the specification detail who
    should perform the step in the supply chain.
  
    The layout describes what each step of the supply chain, along with a set of
    signatures and keys that defines the valid authorizes of the given step.

    TODO: Add an example of a layout

    The link metadata is a JSON file that specifies what the values, and outputs
    of the corresponding step should be. Evaluating the specification along with
    the link metadata lets the users, or the organization, verify that the
    supply chain has not been tampered with. 

    TODO: Add an example of link-metadata

\section{Linux Distributions}\label{sec:linux_distributions}
    \subsection*{Linux}
    Linux is free and open-source project that implements a kernel. It was first
    developed by Linus Torvalds in the early 1990 and has grown into the largest
    open-source project today. It is commonly used in everything from firmware
    modules on a computer, to the every increasing field of Internet of Thing,
    along with servers and on personal computers.

    Linux is accompanied by a suit of tools and environment that is commonly
    referred to as a ''Distribution'' and defines an operating system based on
    Linux. These are created by companies as commercial products, as well as
    groups of volunteers as a hobby for free. A lot of open-source development
    is done with, and for, Linux.

    \subsection*{Debian}
    Debian was one of the first operating systems based on Linux, and was
    created by Ian Murdock in 1993. One of the main innovations from debian was
    the creation of the very first package manager. Package manager allows users
    to download pre-compiled software from centralized repositories maintained
    by the Debian developers.  This allows users to easily fetch, update and
    remove installed packages on their system.

    %Debian has spawned 


\section{Reproducible Builds}\label{sec:reproducible_builds}

    Reproducible builds is a set of practices for how to achieve deterministic
    compilation of software. Most of to

    \say{A build is reproducible if given the same source code, build environment
    and build instructions, any party can recreate bit-by-bit identical copies of
    all specified artifacts.}

    % TODO: Reproducible builds definition

    \subsection*{Determinism}
    \subsection*{Source Date Epoch}
    \subsection*{Buildinfo}

    \subsection*{Rebuilders}%\label{sec:rebuilders} 

\section{Merkle Trees}
Merkle Trees is a tree structure based on cryptographic secure hashing
function.\cite{ralph-c.-merkle-1998} It creates a binary tree where each leaf
is hashed, and combined two and two. The top node of this tree is reffed to as a
''root node''. The interesting property of merkle trees is the ability to verify
the inclusion of elements by calculating the path from the given leaf to the
root node. This can be done by acquiring the missing hashes for each
intermediate node, and then hash each of the steps together.

    % \subsection*{Blockchain}

    \subsubsection*{Certificate transparency log}
    Certificate transparency logs uses merkle trees to give organization issuing
    TLS certificates, as used in HTTP, a log of who and what issues
    log.\cite{b.-laurie-a.-langley-e.kaster-google-2013} This is used to detect
    cases where certificate issue keys have been compromised, or if certificates
    have been issued for domains as a form of misuse.

    Monitors follow these logs to assure consistency and to make sure logs do
    not misbehave. It also allows monitors to verify the append-only property of
    the log.




\section{Development}\label{sec:development} 
    \subsection*{Python}
    \subsection*{sqlalchemy}
    \subsection*{Panda and matplotlib}

\section*{Summary}\label{sec:summary-theory} 

\blankpage
\end{document}
