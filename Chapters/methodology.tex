% !TEX encoding = UTF-8 Unicode
%!TEX root = ../Main/thesis.tex
% !TEX spellcheck = en-US
%%=========================================
\documentclass[../Main/thesis.tex]{subfiles}
\begin{document}
\chapter{Research Methodology}\label{ch:methodology}
This chapter goes through the research methodology, methods and the chosen
framework used in the research. We will address the research questions presented
in section~\ref{sec:rq}.

\section{Design Science Research}%
\label{sec:design_science_research}
In this project we are going to outline a set of needs, requirements and
relevance in the field. This fits very well with the research methodology
``Design Science''.  Design science research is a method that comforts to the
requirements of software development neatly as we are not witnessing events, we
are creating and evaluating them. \citeauthor{Hevner:2004:DSI:2017212.2017217} writes that

\begin{quotation}
``\ldots design-science paradigm has its roots in engineering and the sciences of the
artificial. [\ldots] It seeks to create innovations that define the ideas,
practices, technical capabilities, and products through which the
analysis,design, implementation, management, and use of information systems can
be effectively and efficiently
accomplished''~\cite{Hevner:2004:DSI:2017212.2017217}.
\end{quotation}

In contrast to empirical science, where nature is observed and tested to
understand the observation and gain knowledge, design science is about
identifying the need for something and attempting to create a solution to this
problem. The following evaluation of how well this solution fits the problem at
hand can gives us valuable research in the field of information science and
computer engineering.

\begin{quotation}
``The main research activities involving the natural sciences are to discover how
things are and to justify the reasons for them being so. Natural science
research should be faithful to the observed facts while also being capable of
predicting future observations to some degree''~\cite{Dresch:2014:DSR:2671159}.
\end{quotation}

The result of this research should be an artifact which contributed something to
the field.~\citeauthor{Vaishnavi:2015:DSR:2807332} listed a few examples of
potential contribution Design Science can result in~\cite{Vaishnavi:2015:DSR:2807332}:

\begin{itemize}
    \item Constructs - The conceptual vocabulary of a domain.
    \item Models - Set of propositions or statements expressing relationship
        between constructs.
    \item Frameworks - Real of conceptual guides to serve as support or guide.
    \item Architectures - High-level structures of systems.
    \item Design principles - Core principals and concepts to guide design.
    \item Methods - Sets of steps used to perform tasks.
    \item Instantiations - Situated implementations in certain environments that do or do not
operationalize constructs, models, methods, and other abstract artifacts; in the
latter case such knowledge remains tacit.
    \item Design theories -  A prescriptive set of statements on how to do something to achieve a certain objective
\end{itemize}

In this project the main artifact is an reimplementation of the visualizer
component, as described in~\ref{sec:collab}, with additional improvements.
Following the list from~\citeauthor{Vaishnavi:2015:DSR:2807332} this is an
``instantiation''.

\section{Artifacts}%
\label{sec:artifacts}

\subsection*{Design as an artifact}%
\label{sub:design_as_an_artifact}

\subsection*{Problem relevance}%
\label{sub:problem_relevance}

\subsection*{Design evaluation}%
\label{sub:design_evaluation}

\subsection*{Research evaluation}%
\label{sub:research_evaluation}

\subsection*{Research rigor}%
\label{sub:research_rigor}

\subsection*{Design as a search process}%
\label{sub:design_as_a_search_process}

\subsection*{Communication of research}%
\label{sub:communication_of_research}











\blankpage
\end{document}
