% !TEX encoding = UTF-8 Unicode
%!TEX root = ../Main/thesis.tex
% !TEX spellcheck = en-US
%%=========================================
\documentclass[../Main/thesis.tex]{subfiles}
\begin{document}
\chapter{Research Methodology}\label{ch:methodology}
This chapter goes through the research methodology, methods and the chosen
framework used in the research. We will address the research questions presented
in section~\ref{sec:rq}.

\section{Design Science Research}%
\label{sec:design_science_research}
In this project we are going to outline a set of needs, requirements and
relevance in the field. This fits very well with the research methodology
``Design Science''.  Design science research is a method that comforts to the
requirements of software development neatly as we are not witnessing events, we
are creating and evaluating them. \citeauthor{Hevner:2004:DSI:2017212.2017217} writes that

\begin{quotation}
``\ldots design-science paradigm has its roots in engineering and the sciences of the
artificial. [\ldots] It seeks to create innovations that define the ideas,
practices, technical capabilities, and products through which the
analysis,design, implementation, management, and use of information systems can
be effectively and efficiently
accomplished''~\cite{Hevner:2004:DSI:2017212.2017217}.
\end{quotation}

In contrast to empirical science, where nature is observed and tested to
understand the observation and gain knowledge, design science is about
identifying the need for something and attempting to create a solution to this
problem. The following evaluation of how well this solution fits the problem at
hand can gives us valuable research in the field of information science and
computer engineering.

\begin{quotation}
``The main research activities involving the natural sciences are to discover how
things are and to justify the reasons for them being so. Natural science
research should be faithful to the observed facts while also being capable of
predicting future observations to some degree''~\cite{Dresch:2014:DSR:2671159}.
\end{quotation}

The result of this research should be an artifact which contributed something to
the field.~\citeauthor{Vaishnavi:2015:DSR:2807332} listed a few examples of
potential contribution Design Science can result in~\cite{Vaishnavi:2015:DSR:2807332}:

\begin{itemize}
\label{lst:artifacts}
    \item Constructs - The conceptual vocabulary of a domain.
    \item Models - Set of propositions or statements expressing relationship
        between constructs.
    \item Frameworks - Real of conceptual guides to serve as support or guide.
    \item Architectures - High-level structures of systems.
    \item Design principles - Core principals and concepts to guide design.
    \item Methods - Sets of steps used to perform tasks.
    \item Instantiations - Situated implementations in certain environments that do or do not
operationalize constructs, models, methods, and other abstract artifacts; in the
latter case such knowledge remains tacit.
    \item Design theories -  A prescriptive set of statements on how to do something to achieve a certain objective
\end{itemize}

In this project the main artifact is an reimplementation of the visualizer
component, as described in~\ref{sec:collab}, with additional improvements.
Following the list from~\citeauthor{Vaishnavi:2015:DSR:2807332} this is an
``instantiation''.

\section{Guidelines}%
\label{sec:guidelines}
\citeauthor{Hevner:2004:DSI:2017212.2017217} argues that design science is
inherently a problem solving process. We need to understand the problem, build
a knowledge base and then try come up with a solution to this problem. To aid in
this, they defined a set of seven guidelines to assist researches in
constructive proper design science research projects~\cite{Hevner:2004:DSI:2017212.2017217}.

\subsection*{Design as an artifact}%
\label{sub:design_as_an_artifact}
The main goal of design science research is to produce an artifact. They can
incomplete or complete projects, guidelines or insight into the problem they are
trying to solve~\cite{Hevner:2004:DSI:2017212.2017217}. Example of different artifacts as described by~\citeauthor{Vaishnavi:2015:DSR:2807332} can be found in Section~\ref{lst:artifacts}.

\subsection*{Problem relevance}%
\label{sub:problem_relevance}
The artifact should be relevant to the problem it is trying to
solve.~\citeauthor{Hevner:2004:DSI:2017212.2017217} argues that problems can
formally defined as the difference between the end goal of a state, or and the
current state of things~\cite{Hevner:2004:DSI:2017212.2017217}. Bridging this
gap needs an understanding of the current problem domain to achieve the needed
artifact.

\subsection*{Design evaluation}%
\label{sub:design_evaluation}
The artifact needs to be tested and evaluated to figure out how well it solves
the problem at hand~\cite{Hevner:2004:DSI:2017212.2017217}. This is important to
provide rigor in the research. It also provides assurance of the high quality
the artifact is suppose to have.

\subsection*{Research contributions}%
\label{sub:research_contributions}
The resulting design science research should result in a contribution to the
research
field~\cite{Hevner:2004:DSI:2017212.2017217}.~\citeauthor{Hevner:2004:DSI:2017212.2017217}
says one of the following contributions needs to be founds in a research
project;

\begin{enumerate}
    \item The Design Artifact
    \item Foundation
    \item Methodologies
\end{enumerate}

In our case the artifact is an``Instantiation'', thus the ``Design Artifact'' is
the research contribution. The end goal is to take two separate concepts; the
rebuilder and the transparency log, and attempt to combine them to see if we
gain any additional security guarantees.

\subsection*{Research rigor}%
\label{sub:research_rigor}
~\citeauthor{Hevner:2004:DSI:2017212.2017217} says that ``rigor is achieved by
appropriately applying existing foundations and
methodologies''~\cite{Hevner:2004:DSI:2017212.2017217}. This research is based
on seminal work done by previous research, along with the important work done by
Free- and Open-Source projects in this field. 

\subsection*{Design as a search process}%
\label{sub:design_as_a_search_process}
Design science is suppose to be an iterative process. The goal is to find
effective solutions to the problem at hand. This is done by iterating on
previous implementations until something that solves the problem is
reached~\cite{Hevner:2004:DSI:2017212.2017217}.~\citeauthor{Hevner:2004:DSI:2017212.2017217}
writes that solving problems needs means to reach the desired ends while
satisfying laws from the environment.

\subsection*{Communication of research}%
\label{sub:communication_of_research}
The resulting research needs to be presented by both technologically-inclined
people, as well as people in the business setting. This is important to convey
the importance of the research, as well as aiding future researchers to build
off on the research done in this project ~\cite{Hevner:2004:DSI:2017212.2017217}.


\section{Evaluation}%
\label{sec:methodology_evaluation}
Evaluation is a very important part of design science research. Because our
artifact is an Instantiation we need to decide on a way to test the actual code
base being produced in this thesis.

\citeauthor{Peffers:2012:DSR:2342209.2342243}
in~\citetitle{Peffers:2012:DSR:2342209.2342243} does a literature review on 148
design science articles. They do an open coding approach, where they categorize
the articles using codes, and compared the results across the different
fields~\cite{Peffers:2012:DSR:2342209.2342243}. Their review isn't very
detailed, as noted in their conclusion. But it pinpoints some trends among
engineering fields where design science artifacts, such as instantiations,
are usually evaluated with technical experiments.

They explain technical experiments as;
\begin{quotation}
    ``...a performance evaluation of an algorithm implementation using real-world
    data, synthetic data, or no data, designed to evaluate the technical
    performance, rather than its performance in relation to the real world.
    ''~\cite[p.~402]{Peffers:2012:DSR:2342209.2342243}
\end{quotation}

To evaluate our artifact, we are essentially going to take a look at the package
builds debian does, and see if we are capable of processing them with our
system. This gives us some real world data to take a look at, and help us figure
out if we are on the correct track regarding out system. We will also be writing
up a few APIs. 

APIs are important as they are utilized by developers, however there are no
clear process on how to evaluate good API design.
\citeauthor{Iyer:2012:EAC:2342209.2342213} outlines this problem
in~\citetitle{Iyer:2012:EAC:2342209.2342213}, and comes up with a stakeholder
analysis to help evaluate API design choices when it comes to design
science~\cite{Iyer:2012:EAC:2342209.2342213}.

More importantly, they detail a set of key attributes which they believe are
important for good modular API designs.  
The outline of these key trais are what we are also going to take a look at~\cite[p.~31]{Iyer:2012:EAC:2342209.2342213};
\begin{description}
    \item[Functionality] - The modules are separated and contain logical groupings.
    \item[Hierarchy] - Modules can be decomposed into sub-modules and internal details does not leak.
    \item[Separation of concerns] - Each module is loosely coupled with other modules.
    \item[Interoperability] - Modules can easily interact with others.
    \item[Resuability] - Modules can be reused in other systems.
\end{description}

This couples with a real-world evaluate when it comes to data will guide us in
the evaluation stage of this research.




\blankpage
\end{document}
