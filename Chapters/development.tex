% !TEX encoding = UTF-8 Unicode
%!TEX root = ../Main/thesis.tex
% !TEX spellcheck = en-US
%%=========================================
\documentclass[../Main/thesis.tex]{subfiles}
\begin{document}
\chapter{Development}\label{ch:development}

\section{Requirements}%
\label{sec:requirements}
To establish requirements for the subsequent re-implementation of the
``visualizer'' component we will take a look at the current implementation, and
how the system works. The overarching requirements is the development and the
resulting project being a contribution to the Open-Source community.

\section{Rebuilder}\label{sec:development_rebuilders}
The purpose of the rebuilder is to watch for new packages, queue them, build the
package in a clean environment to reproduce the package, and then publish this
result so we can query them later when installing packages.

To achieve this we need to fulfill a few requirements:

\begin{enumerate}
    \item \label{itm:published} We need to know when a package is published.
    \item \label{itm:scheduler} Something needs to schedule the new packages.
    \item \label{itm:builder} We need to build the package in a clean environment.
    \item \label{itm:publish} We need to publish results of the built package.
    \item \label{itm:transport} We need to check the results when installing packages.
\end{enumerate}

It is important to remember that this system is only targeted Debian as
supporting is universally would require a lot of engineering effort and handling
of special cases.

\begin{figure}[H]
  \centering
  \begin{sequencediagram}
    \newthread{buildinfo}{buildinfo server}{}
    \newthread{scheduler}{scheduler}{}
    \newthread{redis}{redis}{}
    \newthread{builder}{builder}{}
    \newthread{visualizer}{visualizer}{}
    \begin{call}{scheduler}{NewBuildinfo()}{buildinfo}{}\end{call}

    \begin{call}{scheduler}{rpush}{redis}{}
        \postlevel
    \end{call}

    \prelevel\prelevel
    \setthreadbias{east}

    \begin{call}{builder}{rpop}{redis}{}
        \postlevel
    \end{call}

    \setthreadbias{center}
    \begin{callself}{builder}{build}{}
    \end{callself}
    \begin{messcall}{builder}{publish}{visualizer}{}\end{messcall}
  \end{sequencediagram}
\caption{rebuilder sequence diagram}
\label{lst:rebuilder_sequence_diagram}
\end{figure}

The overlying architecture is displayed in~\ref{lst:rebuilder_sequence_diagram} as
a sequence diagram. It gives a quick overview how the different parts interact
with each other in sequence for one successful package rebuild.


\subsection{buildinfo.debian.net}%
\label{sub:buildinfo_debian_net}
The goal is to rebuild packages released by Debian, but getting this information
directly for a Debian package mirror can be tedious. What we instead do is
relaying on the buildinfo server created by the Debian project to keep track of
all published buildinfo files from built packages. This gives us a canonical
view of all packages built by Debian infrastructure.

% TODO: Add refference to pull request
To utilize this service we need to keep a track of all newly submitted files,
however the current API does not support this. To get around this we submitted a
code change so we would be able to get all files submitted after a given
timestamp. As of writing this code change is not accepted.

The rebuilder system has instead relied on a copy of the server with the code
change included so we are able to monitor new buildinfo files.

\subsection{scheduler}%
\label{sub:scheduler}
The scheduler is a small service which monitors the endpoint and schedules any
new files found from the buildinfo server. Currently it pushes new package files
to redis, which is a very simple key value store, to help schedule the builders.
This enables us to add an arbitrary number of builders. This is important for a
few reasons. It helps scaling the system if its needed, and it also allows to
have builders with different architectures to build packages.

Because of builder constraints the current scheduler does not add builds on
other architectures then ``amd64''.

\subsection{builder}%
\label{sub:builder}
The builder consists of a service that queries redis after new items on a timer.
When new builds are dispatched, the build is done by utilizing the buildinfo
files as provided by the Debian build server. The build are done with the tool
``srebuild''.

``srebuild'' is a Perl script used to build packages in a clean environment.
With this environment the buildinfo is parsed and all missing dependencies are
acquired to recreate the package. The source packages, which contains the source
and the build files needed to build the package, is acquired from a mirror and
the build is done. When the build is done, the results are signed with a
cryptographic key, to verify that the build server did produce the files, and
then published to the visualizer.


\subsection{visualizer}%
\label{sub:visualizer}
The visualizer is the component which displays the rebuilt packages in a web UI.
The user is also able to fetch the buildinfo and linkmetadata files. The current
implementation is a short snippet of code backed by a sqlite database to aid in
displaying the needed webpages. The implemented API as seen in
\ref{api:old_visualizer} is simplistic and provides the needed features to let
users verify builds.


\begin{table}[H]
\footnotesize
\centering
\settowidth\tymin{\textbf{Endpoint}}
\setlength\extrarowheight{2pt}
\begin{tabulary}{\textwidth}{|l|L|l|L|}
\hline
    \textbf{Endpoint} & 
    \textbf{Type} & 
    \textbf{Parameters} & 
    \textbf{Description} \\
\hline
    /new\_build & POST$^1$ & metadata, buildinfo & Submit a new build \\  \hline
    /sources/<name> & GET & & Gets the available builds for a package \\  \hline
    /sources/<name>/<version>/buildinfo& GET & & Gets the BUILDINFO file for a build \\  \hline
    /sources/<name>/<version>/metadata & GET & & Gets the in-toto link metadata from a build \\  \hline
\end{tabulary}
\footnotesize{$^1$ Behind authentication}\\
\caption{Old visualizer API}
\label{api:old_visualizer}
\end{table}


\section{Project development}%
\label{sec:project_development}
We have now taken a look at the current implementation of the rebuilder system,
and how it integrates with the current iteration of the visualizer. In the next
session we will explain the development of the system for this thesis. It's
structured in 3 iterations of the visualizer, and an integration with the
existing APT transport written for the initial rebuilder system. We will in the
first iteration tackle the problem of maintaining compatibility with the current
system. The second iteration will be focusing on the implementation of the raw
merkle tree needed for transparency log. The third iteration will be the
abstracted logic on top of the transparency log, which the APT transport will be
utilizing when validating packages for the users.

\subsection{First Iteration: Visualizer}%
\label{sec:visualizer}

\subsection*{Goals}%
\label{sub:first_iteration_goals}
The first iteration is largely focused on recreating the functionality of the
current visualizer. The purpose of this component is to accept new rebuilds, the
in-toto link metadata file and buildinfo file produced by the rebuilder. This
needs to be displayed in a simple webpage and introduce no new features to
remain compatible.

Thus the goals for this iteration is as follows;

\begin{itemize}
    \item Maintain compatibility with the old visualizer
    \item Accept new build submissions
    \item Display packages with build submissions
    \item Display the submission files given a package name and version
\end{itemize}

\subsection*{Development}%
\label{sub:first_iteration_development}
The initial task for this project was figuring out a structure that was flexible
and made sense for the further development. The structure that was aimed upon
was to separate database models in its own directory, templates in its own
directory and views in it's own. When this setup was done we could continue with
the development of the visualizer.

In practice the visualizer only accepts two files, and displays an index of
these files. There are no processing being done except to figure the given
package name and version. The goal of this rewrite is to create a robust
foundation where we can improve on the current design, and in later iterations
build the needed data structures.


\begin{figure}[H]
\centering
\begin{tikzpicture}[
    EMP/.style={% Style for empatized boxes
        rectangle, line width =1pt,
        anchor=west,
        underline, % new property
        align=center,
        text=black,
        minimum height=.8cm,
        text height=1.5ex,
            text depth=.25ex,
        fill=EMP,
        draw=black,
        },
    NOR/.style={% Style for normal boxes.
        rectangle, 
        line width =1pt,
        anchor=west,
        align=left,
        minimum height=.6cm,
        text height=1.5ex,
            text depth=.25ex,
            text=white,
        fill=NOR,
        draw=black,
        inner ysep=5pt
        },
    underline/.append style={% define new style property
        execute at begin node={%
            \setbox\ubox=\hbox\bgroup
            },
            execute at end node={%
                \egroup\uline{\box\ubox}%
                }
             },
    ] % Uff that is all the configuration for tickzpicture xD

 \def\Frame(#1)#2[#3]#4{%
  \begin{scope}[shift={(#1)}] 
      \node[font=\bf, anchor=west] (Title) at (-0.2,0.7) {#3}; 
       \edef\k{0}
       \edef\x{0}% Variable for named coordinate centering - below box
       \foreach \id/\style in {#4} {%enter sub frame data Name/Boxtype ,Name2/Boxtype | An space before Boxtype is needed 
            \node[\style] (h) at (\k pt,0) {\id}; %  % Draw a node depending on the variables.
            \pgfmathparse{\k+0.5*width{"\id"}+3.4pt} % Uses the textwidth to calculate named coordinate  
            \xdef\x{\pgfmathresult} % The resul is saved in the variable \x
            \draw (\x pt,-0.4) coordinate (\id#2); %Create a named coordinate concatenated: "sub frame data Name"+"identifier"
            \pgfmathparse{\k+width{"\id"}+6.8pt}% Calculate positión for each subframe box.       
        \xdef\k{\pgfmathresult}% Save the value to be added to the next iteration value.
       }    
  \end{scope}
}
 \Frame(0,0){1}[BUILDINFO]{%first frame identified as 1 named EMPLOYEE
    Id/NOR,% see that it is necessary to add a space
    Created/NOR,
    Text/NOR,
    UUID/NOR,
    VersionId/EMP}; 

 \Frame(0,-2.5){2}[LINKMETADATA]{
    Id/NOR,
    Created/NOR,
    Text/NOR,
    UUID/NOR,
    VersionId/EMP}; 

 \Frame(0,-5){3}[VERSION]{
    Id/NOR,
    Created/NOR,
    Version/NOR,
    PackageId/EMP};

  \Frame(0,-7.5){4}[PACKAGE]{
    Id/NOR,
    Created/NOR,
    Name/NOR}; 

     \draw[thick,->,thick,>=latex]
        (PackageId3) -- ++(0,-.5) -- ++(0,0) coordinate (inter) 
        -- (Id4 -| inter) -- ++(0,-0.4) coordinate (inter)
        -- (Id4 |- inter) -- ++(0,0.5); %

     \draw[thick,->,thick,>=latex]
        (VersionId2) -- ++(0,-0.85) -- ++(.8,0) coordinate (inter) 
        -- (Id3 -| inter) -- ++(0,-0.2) coordinate (inter) 
        -- (Id3 |- inter) -- ++(0,0.3); %

     \draw[thick,->,thick,>=latex]
        (VersionId1) -- ++(0,-0.85) -- ++(1.5,0) coordinate (inter) 
        -- (Id3 -| inter) -- ++(0,-0.4) coordinate (inter) 
        -- (Id3 |- inter) -- ++(-.15,0) -- ++(0,0.5); %

\end{tikzpicture}
\caption{Database schema}
\label{fig:schema}
\end{figure}

The first step is to make sure the database format is correctly represented. The
previous iteration had a strict dependency on sqlite, which is a very simple
database stored in a single file. This works well for point of concept
implementations and where the database does not grow exceedingly large.

In the rewrite we will be utilizing an ORM for python, the sqlalchemy library.
This will allow us to define data models and instantiate them on top of
different database engines. The database structure itself closely copies the one
from the original implementation. The schema as displayed on~\ref{fig:schema},
page~\pageref{fig:schema}, represent the implemented model in the ORM.

The schema implements the model as follows; ``package'' can have multiple
``versions''. The models that belong to ``linkmetadata`` and ``buildinfo`` is
the tables containing the data itself. In the previous iteration these where
stored as plain text files, which is a less portable way of dealing with the
data. There can be multiple submissions for each version, so this relation is a
``One-to-Many`` relationship where one ``version'' can have multiple submissions
from rebuilders.

The ``UUID'' field found in the ``linkmetadata'' and ``buildinfo'' model is
mostly a hack. The main issue was to find the pairs of submissions without
over-complicating the database structure. One solution would be to create a new
table to associate the submissions. This would enable us to properly group them
later on an find the individual pairs. However, because of some time
constraints, and because the implementation of a new model would take some time,
the addition of an unique ``UUID'' for each submission is an easier alternative
that is simple to implement. This allows us to group the submissions for the
frontend later on.

It should be noted that the ORM handles the One-to-many and many-to-many
relationships. They are not explicitly included in the modeled schema for on 
for the sake of brevity.

\begin{figure}[H]
\begin{minted}[]{python}
class Version(db.Model):
    __tablename__ = "version"
    id = db.Column(
            db.Integer(),
            index=True, unique=True, 
            primary_key=True, autoincrement=True)

    created = db.Column(db.DateTime, nullable=False, default=datetime.utcnow)
    version = db.Column(db.String(64), nullable=False)

    buildinfo = db.relationship("Buildinfo", back_populates="version")
    linkmetadata = db.relationship("LinkMetadata", back_populates="version")

    package_id = db.Column(db.Integer, db.ForeignKey("package.id"))
    package = db.relationship("Package", back_populates="version")

    def __repr__(self):
        return "<Version: {}>".format(self.version)
\end{minted}
\caption{Sqlalchemy code for the Version model}
\label{fig:version-model}
\end{figure}

The sqlalchemy library enables us to define database models as native Python
classes. As can be seen in~\ref{fig:version-model} we are able to represent the
needed files, and relationships for the ``Version'' model. In total the
implementation consists of 4 classes which we are able to query, create and
update. This lets us avoid having to implement the transition from the database
format to the correct data representation in the project.

\subsection*{API and Frontend}%
\label{sub:api_and_frontend}

To reimplement the API as can be seen in table~\ref{api:old_visualizer} on
page~\pageref{api:old_visualizer}, we need to create the needed routes for the
web service. This is being done by Flask, which is a webserver framework for
python. The same framework was utilized in the original implementation and
enables us to mainly substitute the code from the old database implementation,
to the one utilizing the ORM. There are two parts to this, one part needs to
return the plain-text elements for the APT transport, and one part needs to
render HTML webpages for the user too browse.


\begin{figure}[H]
\begin{minted}[]{python}
@app.route("/sources/<pkgname>")
def all_sources_pkg(pkgname):
    entries = (
        db.session.query(Package, Version, LinkMetadata, Buildinfo)
        .join(Version, Version.package_id == Package.id)
        .filter(Package.name == pkgname)
        .filter(LinkMetadata.version_id == Version.id)
        .filter(LinkMetadata.uuid == Buildinfo.uuid)
    ).all()
    return render_template("source.html", package=pkgname, entries=entries)
\end{minted}
\caption{Python code for source.html}
\label{fig:python-source}
\end{figure}

The figure seen in~\ref{fig:python-source} is a simple example of a route
written in Python. It calls out to the needed models, and does an implicit join
between them to get the needed relationships in place. The results of this is a
webpage containing all the rebuilder submissions from the builder. Similar
endpoints are written to provide rest of the functionality as described
in~\ref{api:old_visualizer}. The input for this route is the name of the package
and the version of the given package.


\begin{figure}[H]
\begin{minted}[]{jinja}
<!DOCTYPE html>
<html>
  <head><title> {{package}} </title></head>
  <body>
    <table>
      <tr>
        <th> Version </th>
        <th> Timestamp </th>
        <th> Buildinfo </th>
        <th> in-toto metadata </th>
      </tr>
      
      <tr>
        <td> {{package}}-{{entry[0].version}}</td>
        <td> {{entry[1].created }} </td>
        <td> 
            <a href="/sources/{{package}}/{{entry[0].version}}/buildinfo">Link</a>
        </td>
        <td> 
            <a href="/sources/{{package}}/{{entry[0].version}}/metadata">Link</a>
        </td>
      </tr>
      
    </table>
  </body>
</html>
\end{minted}
\caption{jinja2 template for source.html}
\label{fig:jinja-source}
\end{figure}

To display these results, we are utilizing jinja2 templates that lets us compose
HTML and Python code to generate webpages on the fly. The code as shown on
Figure~\ref{fig:jinja-source} is how we render the results of
Figure~\ref{fig:python-source} which contains all the submissions given a
package and version.

\subsection*{Results}%
\label{sub:first_iteration_results}

The result of this development is a mirror of the previous visualizer
implementation. The main requirement is to be able to query the last submitted
buildinfo and linkmetadata file. The fact that it only is capable of querying
the last submission is a design choice from the first revisions as it only
stored the last submitted files. For compatibility reasons this is kept in the
rewrite.

\begin{figure}[H]
\center{\includegraphics[]{../Pictures/overview.png}}
\caption{Overview of rebuild packages}%
\label{fig:rebuild-overview} 
\end{figure}

\begin{figure}[H]
\center{\includegraphics[width=\textwidth]{../Pictures/sources-view.png}}
\caption{Overview of the rebuild submissions}%
\label{fig:submission-overview} 
\end{figure}

The website as shown on Figure~\ref{fig:rebuild-overview} and
Figure~\ref{fig:submission-overview} displays the finished webserver for the
initial rewrite of the visualizer. The resulting code and webpage maintains
compatibility of the existing visualizer and enables us to continue implementing
further improvements on the API.

\subsection{Second Iteration: Merkle Tree}%
\label{sub:merkle_tree}
The second iteration focuses on implementing merkle trees into the visualizer.
These will create the foundation of the transparency log in the upcoming
iterations for this project.

These will be implemented along side of the reimplementation of the visualizer,
and the current structure of the code. We will take a look at the challenges of
implementing this correctly along with making sure the proofs are correctly
implemented to support a transparency log.

\subsection*{Goals}%
\label{sub:second_iteration_goals}
The goals of this iteration is as to make sure the merkle tree operated
properly. For this to happen there needs to be validated proofs and a proper
tree generated when new items are added.

\begin{itemize}
    \item Implement the needed database models for the datastructure
    \item New items should be appended to the tree
    \item The tree should not be rebuilt for each append
    \item Audit proofs should be implemented
    \item Consistency proofs should be implemented
    \item Graph the tree
\end{itemize}


\subsection*{Development}%
\label{sub:second_iteration_development}
The main challenge in this iteration is getting the underlying data model
correct. What we are trying to achieve is to implement a tree structure where
the hash of the leafs is hashed together to form a chain of checksums all the
way up to the tree root. The following algorithm to achieve this is described
below.

\begin{enumerate}
    \item If there is an odd number of nodes, take the last node and promote to
        next iteration
    \item Pop two nodes from the current level
    \item Hash both nodes, concatenate them and append node to next iteration
    \item Repeat from 1.
\end{enumerate}

Along with making sure the elements are hashed together, we need to keep some
track of which node is the child of which nodes as we do need to traverse
this tree later when reconstructing audit and consistency paths. The 


\subsection*{Database}%
\label{sub:database}
Since we have implemented the visualizer with ``sqlalchemy'', the idea is to
utilize the SQL backend to store the tree. This gives a few challenges on how to
structure the model when it comes with relationships.

\subsection*{Construct tree}%
\label{sub:construct_tree}



\begin{listing}[H]
\caption{Node hashing strategy}
\label{lst:node_hashing_strategy}
\begin{equation*}
H\{\text{NodeType},\ \text{LeftNode},\ \text{RightNode}\}
\end{equation*}
\end{listing}

\subsection*{Graphviz}%
\label{sub:graph_merkle_tree}
To validate that we are getting the correct association and relationships when
appending new elements, one of the important things to get is a visualization of
the currently constructed merkle tree. To aid the discovery of problematic nodes
the first 10 letters of the hash is printed with the node to easily find the
affected node and aid debugging.

To do this we loop through all of the nodes in the tree. We print out the left,
and right parent of the node in a format that is compatible with the graphviz
tool to generate pictures. The output found in~\ref{lst:graphviz-test} enables
us to produce the PDF picture in~\ref{fig:graphviz_pdf} by supplying the output
to ``dot -Tpdf''.

It should be noted that when trees get sufficiently large, the graph it self is
of less value when debugging. It is hard to get a good overview when the tree
reaches around 60 nodes as the generated trees are spacious. Not enough time was
invested trying to fiddle with the style got a proper display when the trees
grow large enough. However, they did aid in early debugging when trees where
sparse.

\begin{listing}[htpb]
\caption{Example graph of a generated tree}
\label{lst:graphviz-test}
\begin{minted}{bash}
$ curl 127.0.0.1:5000/api/log/tree/graphviz                        
graph graphname {
labelloc="t";
label="Nodes: 4";
"9A565DAB05" -- "B439C9F4B0";
"9A565DAB05" -- "730212F389";
"B439C9F4B0" -- "47A2AB3D2E";
"B439C9F4B0" -- "648D1817A3";
"730212F389" -- "92EF30AD83";
"730212F389" -- "1E94F9FDE4";
}
\end{minted}
\end{listing}

\begin{figure}[htpb]
\centering
\includegraphics[width=\textwidth]{../Diagrams/graph.pdf}
\caption{Graphviz visualization}
\label{fig:graphviz_pdf}
\end{figure}

\subsection*{Audit proof}%
\label{sub:audit_proof_implementation}

\subsection*{Consistency proof}%
\label{sub:consistency_proof_implementation}

\subsection*{Consistency proof}%
\label{sub:consistency_proof_implementation}

\begin{listing}[H]
\begin{minted}{python}
class Node(db.Model):
    __tablename__ = "node"
    id = db.Column(
        db.Integer(), index=True, unique=True, primary_key=True, autoincrement=True
    )
    leaf_index = db.Column(db.Integer(), default=0)
    type = db.Column(db.String(10), nullable=False)
    hash = db.Column(db.String(128))
    created = db.Column(db.DateTime, nullable=False, default=datetime.utcnow)
    data = db.Column(JSONB)
    height = db.Column(db.Integer(), default=0)
    children_right_id = db.Column(db.Integer, db.ForeignKey("node.id"))
    children_left_id = db.Column(db.Integer, db.ForeignKey("node.id"))

    right = db.relationship("Node",
            foreign_keys='Node.children_right_id',
            uselist=False,
            backref=db.backref('children_right', remote_side=[id]))

    left = db.relationship("Node", 
            foreign_keys='Node.children_left_id',
            uselist=False,
            backref=db.backref('children_left', remote_side=[id]))

    children_id = db.Column(db.Integer, db.ForeignKey("node.id"))
\end{minted}
\caption{Node sqlalchemy model}
\label{lst:node_model}
\end{listing}

\subsection*{Results}%
\label{sub:second_iteration_results}


\begin{table}[H]
\footnotesize
\centering
\settowidth\tymin{\textbf{Description}}
\setlength\extrarowheight{2pt}
\begin{tabulary}{1.0\textwidth}{|l|L|l|L|}
\hline
    \textbf{Endpoint} & 
    \textbf{Type} & 
    \textbf{Parameters} & 
    \textbf{Description} \\
\hline
    /api/log/stats & GET & & Statistics from the current tree \\ \hline
    /api/log/graphviz & GET & & Outputs the current tree in the dot format \\ \hline
    /api/log/root & GET & & Gets the latest tree root \\ \hline
    /api/log/tree/append & POST & JSON object & Appends a JSON object to the tree \\ \hline
    /api/log/tree/id/<id> & GET & & Gets the given Node object from the database with '``id''\\ \hline
    /api/log/tree/hash/<hash> & GET & & Gets the given Node object from the database with ``hash''\\ \hline
    /api/log/tree/leaf/<id> & GET & & Gets the given leaf from the database with matching ``leaf\_index''\\ \hline
    /api/log/tree/validate/id/<id> & GET & & Gets the audit proof for a leaf matching the given ``id''\\ \hline
    /api/log/tree/validate/hash/<hash> & GET & & Gets the audit proof for a leaf matching the given ``hash''\\ \hline
    %/api/log/tree/inclusion & POST & InclusionQuery & Provides the inclusion proof for the given query \\ \hline
    /api/log/tree/consistency & POST & ConsistencyQuery & Provides the consistency proof for the given query\\ \hline
\end{tabulary}
\caption{Second iteration: Transparency log API}
\label{api:transparency_log}
\end{table}


\begin{listing}[H]
\caption{JSON for consistency proof}
\label{lst:consistency proof}
\begin{minted}{json}
{}
\end{minted}
\end{listing}

\begin{listing}[H]
\caption{JSON for audit proof}
\label{lst:audit proof}
\begin{minted}{json}
{}
\end{minted}
\end{listing}


\subsection{Third Iteration: Tree root signing}%
\label{sub:tree_root_signing}
One essential part of merkle trees as implemented
by~\citeauthor{b.-laurie-a.-langley-e.kaster-google-2013} in certificate
transparency logs is the ability to have tree-roots
signed~\cite{b.-laurie-a.-langley-e.kaster-google-2013}. This lets us verify
with public key cryptography that the tree-root was created by the given
transparency log. In this chapter we will go through how this implemented.

\subsection*{Goals}%
\label{sub:third_iteration_goals}

The goals of this iteration is to create the needed code to create and verify
signatures on tree nodes.

\begin{itemize}
    \item Initialize signing keys
    \item Sign every tree root created by the log
    \item Verify the signature
\end{itemize}

\subsection*{Development}%
\label{sub:third_iteration_development}
For the development of this feature we are utilizing the ``securesystemslib'' from
New York University's' secure systems lab. It's a library with a collection of
easy-to-use primitives to deal with encryption, decryption and signature
verification on data for python. Utilizing this library makes it trivial to
implement the needed signature functionality in a short concise manner.

\begin{listing}[H]
\caption{Glue code for ``securesystemslib''}
\label{lst:securesystemslib_glue}
\begin{minted}{python}
PASSWORD = "123"
KEY_NAME = "ed25519_key"

def init_keys():
    if not os.path.isfile(KEY_NAME):
        generate_and_write_ed25519_keypair(KEY_NAME, password=PASSWORD)

def get_private_key():
    if not os.path.isfile(KEY_NAME):
        init_keys()
    return import_ed25519_privatekey_from_file(KEY_NAME, password=PASSWORD)

def get_public_key():
    if not os.path.isfile(KEY_NAME):
        init_keys()
    return import_ed25519_publickey_from_file(KEY_NAME+'.pub')

def sign_data(data):
    return create_signature(get_private_key(), data)

def verify_data(signature, data):
    return verify_signature(get_private_key(), signature, data)
\end{minted}
\end{listing}

The implementation itself consists of a file for the needed code to ease the key
creation for the rebuilder. For the sake of ease, we don't consider problems
such as password strength for the signing key as the main focus is on the
transparency log implementation, so in this implementation the password is
hard coded to ``123'', which is admittedly a poor password. For the
cryptography, elliptic curves are used instead of the traditional RSA algorithm.
The choice behind this is because elliptic curves needs less bits to produce
equally strong keys. We then end up with faster and smaller signatures.

These auxiliary functions makes it trivial to further implement the tree root
signing in the model.

\begin{listing}[H]
\caption{Additions to the Node model}
\label{lst:node_tree_root_signature}
\begin{minted}{diff}
@@ -21,10 +22,10 @@ class Node(db.Model):
     leaf_index = db.Column(db.Integer(), default=0)
     type = db.Column(db.String(10), nullable=False)
     hash = db.Column(db.String(128))
+    signature = db.Column(db.String(128))
     created = db.Column(db.DateTime, nullable=False, default=datetime.utcnow)
     data = db.Column(JSONB)
     height = db.Column(db.Integer(), default=0)
\end{minted}
\end{listing}

In Listing~\ref{lst:node_tree_root_signature} we can see the additions to the
``Node'' model from Listing~\ref{lst:node_model} page~\pageref{lst:node_model}.
This enables us to store the signature string in the database model for future
use when creating tree roots.

\begin{listing}[H]
\caption{Additions to the append function}
\label{lst:append_tree_root_signing}
\begin{minted}{diff}
@@ -166,7 +168,9 @@ def append(data):
     for node in reversed(subtrees):
         new_parent = create_level_node(node, new_node)
         new_node = new_parent
+    signature = sign_data(new_node.hash)
+    new_node.signature = signature["sig"]
     db.session.commit()
     return ret
\end{minted}
\end{listing}

Listing~\ref{lst:append_tree_root_signing} shows the additions made to the
``append'' function to support signed tree roots. Since the last ``new\_node'' is
the new tree root, we only sign the hash of this node. This is enough to
implement tree root signing in the current implementation of the visualizer.

\subsection*{Results}%
\label{sub:third_iteration_results}

The results of this iteration is 2 new endpoints and some new code to help with
signing tree roots in the transparency log. The resulting API endpoints can be
found in Table~\ref{api:crypto_api} which shows the newly created endpoints.

\begin{table}[H]
\footnotesize
\centering
\settowidth\tymin{\textbf{Description}}
\setlength\extrarowheight{2pt}
\begin{tabulary}{1.0\textwidth}{|l|L|l|L|}
\hline
    \textbf{Endpoint} & 
    \textbf{Type} & 
    \textbf{Parameters} & 
    \textbf{Description} \\
\hline
    /api/crypto/key & GET & & Outputs the key object used by the library \\ \hline
    /api/crypto/validate & POST & Tree root & Validates the tree root to the public key \\ \hline
\end{tabulary}
\caption{Third Iteration: Crypto API }
\label{api:crypto_api}
\end{table}

The usage of the endpoints is fairly straight forward. We can call the generated
signing key with ``/api/crypto/key'' as shown
in~\ref{lst:display-generated-public-key}. Which enables us to verify the
signature on the client side instead of on the server side if we need to do
that.

\begin{listing}[H]
\caption{Display generated public key}
\label{lst:display-generated-public-key}
\begin{minted}{bash}
$ curl 127.0.0.1:5000/api/key
{
  "keytype": "ed25519",
  "scheme": "ed25519",
  "keyid": "25a16bb3a3...",
  "keyid_hash_algorithms": [
    "sha256",
    "sha512"
  ],
  "keyval": {
    "public": "7623ba359c..."
  }
}
\end{minted}
\end{listing}

To verify a tree root on the server side, we can provide the output from
``/api/log/tree/root'' from Listing~\ref{api:transparency_log}, and forward this
to ``/api/crypto/verify''. In the example from
Listing~\ref{lst:test-verify-endpoint}, we have provided a minimal tree root
json object to display a successful verification of a tree root on the server
side.

\begin{listing}[H]
\caption{Test of the verify endpoint}
\label{lst:test-verify-endpoint}
\begin{minted}{bash}
$ curl --header "Content-Type: application/json" \
  --request POST \
  --data '{"hash": "21e0b13a6f...", \
           "signature": "832510c0188..."}' \
  127.0.0.1:5000/api/crypto/verify
{
  "status": "ok",
  "verified": true
}
\end{minted}
\end{listing}


\subsection{Fourth Iteration: Transparency log overlay}%
\label{sub:transparency_overlay}

\subsection*{Goals}%
\label{sub:fourth_iteration_goals}

\subsection*{Development}%
\label{sub:fourth_iteration_development}

API entry definitions

\begin{listing}[H]
\caption{Entry definitions}
\label{lst:entrydefinitions}
\begin{minted}{go}
type Entry {
    Package string
    Version string
} 

type InclusionEntry {
    Entry
    Buildinfo       string
    Linkmetadata    string
}

type RevokeEntry {
    Entry 
    InclusionHash   string
    Reason          string
}
\end{minted}
\end{listing}


\begin{table}[hbtp]
\footnotesize
\centering
\settowidth\tymin{\textbf{Description}}
\setlength\extrarowheight{2pt}
\begin{tabulary}{1.0\textwidth}{|l|L|L|L|}
\hline
    \textbf{Endpoint} & 
    \textbf{Type} & 
    \textbf{Parameters} & 
    \textbf{Description} \\
\hline
    /api/rebuilder/submit & POST$^1$ & metadata, buildinfo& Implements something heyho lets go \\  \hline
    /api/rebuilder/revoke& POST$^1$ & leaf hash, reason, signature& Revokes the leaf with a reason  \\  \hline
    /api/rebuilder/fetch/<name>/<version>& GET & & Gets the entries for the given package \\  \hline
\end{tabulary}
\footnotesize{$^1$ Behind authentication}\\
\caption{Overlay API}
\label{api:Overlay API}
\end{table}

\subsection*{Results}%
\label{sub:fourth_iteration_results}


\subsection{APT Transport integration}\label{sec:apt_transport}

\begin{figure}[H]
  \centering
  \begin{sequencediagram}
    \newthread{A}{HTTP client}{}
    \newinst[1]{B}{intoto}{}
    \newinst[2]{C}{APT Server}{}
    \begin{messcall}{A}{100 Capabilities}{C}
        \begin{messcall}{C}{601 Configuration}{A}\end{messcall}
        \begin{messcall}{C}{600 URI Acquire}{A}\end{messcall}
        \begin{messcall}{A}{200 URI Start}{C}\end{messcall}
            \begin{messcall}{C}{201 URI Done}{B}
                \begin{sdblock}{Rebuilder verify}{}
                    \begin{messcall}{B}{201 URI Done}{A}\end{messcall}
                \end{sdblock}
            \end{messcall}
    \end{messcall}
  \end{sequencediagram}
\caption{intoto sequence diagram}
\label{lst:intoto_sequence_diagram}
\end{figure}

\begin{listing}[H]
\caption{Initial API request}
\label{lst:init_request}
\begin{align*}
& \text{InitQuery}\{\text{Hash},\ \text{Signature},\ \text{LeafCount}\} \to \\
& \text{Response}\{\text{ConsistencyProof},\ \text{CurrentRoot}\{\text{Hash},\ \text{Signature}\} \}
\end{align*}
\end{listing}


\begin{listing}[H]
\caption{Package query API request}
\label{lst:package_query_request}
\begin{equation*}
    \text{PkgQuery}\{\text{Package},\ \text{Version}\} \to \text{Response}\{\text{InclusionEntry},\ \text{Entry}\dots \}
\end{equation*}
\end{listing}


\section{Testing}%
\label{sec:testing}



\blankpage
\end{document}
