% !TEX encoding = UTF-8 Unicode
%!TEX root = ../Main/thesis.tex
% !TEX spellcheck = en-US
%%=========================================
\documentclass[../Main/thesis.tex]{subfiles}
\begin{document}
\chapter{Development}\label{ch:development}

\section{Requirements}%
\label{sec:requirements}
To establish requirements for the subsequent re-implementation of the
``visualizer'' component we will take a look at the current implementation, and
how the system works. The overarching requirements is the development and the
resulting project being a contribution to the Open-Source community.

\section{Rebuilder}\label{sec:development_rebuilders}
The purpose of the rebuilder is to watch for new packages, queue them, build the
package in a clean environment to reproduce the package, and then publish this
result so we can query them later when installing packages.

To achieve this we need to fulfill a few requirements:

\begin{enumerate}
    \item \label{itm:published} We need to know when a package is published.
    \item \label{itm:scheduler} Something needs to schedule the new packages.
    \item \label{itm:builder} We need to build the package in a clean environment.
    \item \label{itm:publish} We need to publish results of the built package.
    \item \label{itm:transport} We need to check the results when installing packages.
\end{enumerate}

It is important to remember that this system is only targeted Debian as
supporting is universally would require a lot of engineering effort and handling
of special cases.

\begin{figure}[H]
\centering
\includegraphics[width=0.5\textwidth]{../Diagrams/architecture.pdf}
\caption{Rebuilder architecture overview}
\label{fig:rebuilder_architecture}
\end{figure}

\begin{figure}[H]
  \centering
  \begin{sequencediagram}
    \newthread{buildinfo}{buildinfo server}{}
    \newthread{scheduler}{scheduler}{}
    \newthread{redis}{redis}{}
    \newthread{builder}{builder}{}
    \newthread{visualizer}{visualizer}{}
    \begin{call}{scheduler}{NewBuildinfo()}{buildinfo}{}\end{call}

    \begin{call}{scheduler}{rpush}{redis}{}
        \postlevel
    \end{call}

    \prelevel\prelevel
    \setthreadbias{east}

    \begin{call}{builder}{rpop}{redis}{}
        \postlevel
    \end{call}

    \setthreadbias{center}
    \begin{callself}{builder}{build}{}
    \end{callself}
    \begin{messcall}{builder}{publish}{visualizer}{}\end{messcall}
  \end{sequencediagram}
\caption{rebuilder sequence diagram}
\label{lst:rebuilder_sequence_diagram}
\end{figure}


\subsection{buildinfo.debian.net}%
\label{sub:buildinfo_debian_net}
The goal is to rebuild packages released by Debian, but getting this information
directly for a Debian package mirror can be tedious. What we instead do is
relaying on the buildinfo server created by the Debian project to keep track of
all published buildinfo files from built packages. This gives us a canonical
view of all packages built by debian infrastructure.

% TODO: Add refference to pull request
To utilize this service we need to keep a track of all newly submitted files,
however the current API does not support this. To get around this we submitted a
code change so we would be able to get all files submitted after a given
timestamp. As of writing this code change is not accepted.

The rebuilder system has instead relied on a copy of the server with the code
change included so we are able to monitor new buildinfo files.

\subsection{scheduler}%
\label{sub:scheduler}
The scheduler is a small service which monitors the endpoint and schedules any
new files found from the buildinfo server. Currently it pushes new package files
to redis, which is a very simple key value store, to help schedule the builders.
This enables us to add an arbitrary number of builders. This is important for a
few reasons. It helps scaling the system if its needed, and it also allows to
have builders with different architectures to build packages.

Because of builder constraints the current scheduler does not add builds on
other architectures then ``amd64''.

\subsection{builder}%
\label{sub:builder}
The builder consists of a service that queries redis after new items on a timer.
When new builds are dispatched, the build is done by utilizing the buildinfo
files as provided by the Debian build server. The build are done with the tool
``srebuild''.

``srebuild'' is a Perl script used to build packages in a clean environment.
With this environment the buildinfo is parsed and all missing dependencies are
acquired to recreate the package. The source packages, which contains the source
and the build files needed to build the package, is acquired from a mirror and
the build is done. When the build is done, the results are signed with a
cryptographic key, to verify that the build server did produce the files, and
then published to the visualizer.


\subsection{visualizer}%
\label{sub:visualizer}
The visualizer is the component which displays the rebuilt packages in a web UI.
The user is also able to fetch the buildinfo and linkmetadata files. The current
implementation is a short snippet of code backed by a sqlite database to aid in
displaying the needed webpages. The implemented API as seen in
\ref{api:old_visualizer} is simplistic and provides the needed features to let
users verify builds.

\begin{table}[H]
\footnotesize
\centering
\settowidth\tymin{\textbf{Endpoint}}
\setlength\extrarowheight{2pt}
\begin{tabulary}{\textwidth}{|l|L|l|L|}
\hline
    \textbf{Endpoint} & 
    \textbf{Type} & 
    \textbf{Parameters} & 
    \textbf{Description} \\
\hline
    /new\_build & POST & metadata, buildinfo & Submit a new build \\  \hline
    /sources/<name> & GET & & Gets the available builds for a package \\  \hline
    /sources/<name>/<version>/buildinfo& GET & & Gets the BUILDINFO file for a build \\  \hline
    /sources/<name>/<version>/metadata & GET & & Gets the in-toto link metadata from a build \\  \hline
\end{tabulary}
\caption{Old visualizer API}
\label{api:old_visualizer}
\end{table}


\section{Project development}%
\label{sec:project_development}


API entry definitions

\begin{listing}[H]
\caption{Entry definitions}
\label{lst:entrydefinitions}
\begin{minted}{go}
type Entry {
    Package string
    Version string
} 

type InclusionEntry {
    Entry
    Package         string
    Buildinfo       string
    Linkmetadata    string
}

type RevokeEntry {
    Entry 
    InclusionHash   string
    Reason          string
}
\end{minted}
\end{listing}


\begin{listing}[H]
\caption{Initial API request}
\label{lst:init_request}
\begin{align*}
& InitQuery\{Hash, Signature, LeafCount\} \to \\
& Response\{InclustionProof, ConsistencyProof, CurrentRoot\{ Hash, Signature\} \}
\end{align*}
\end{listing}


\begin{listing}[H]
\caption{Package query API request}
\label{lst:package_query_request}
\begin{equation*}
PkgQuery\{Package, Version\} \to Response\{InclusionEntry, Entry\dots \}
\end{equation*}
\end{listing}


\begin{listing}[H]
\caption{JSON for consistency proof}
\label{lst:consistency proof}
\begin{minted}{json}
{}
\end{minted}
\end{listing}


\begin{listing}[H]
\caption{JSON for inclusion proof}
\label{lst:inclusion proof}
\begin{minted}{json}
{}
\end{minted}
\end{listing}


\begin{listing}[H]
\caption{JSON for audit proof}
\label{lst:audit proof}
\begin{minted}{json}
{}
\end{minted}
\end{listing}


\section{Visualizer}\label{sec:visualizer}




\begin{table}[hbtp]
\footnotesize
\centering
\settowidth\tymin{\textbf{Description}}
\setlength\extrarowheight{2pt}
\begin{tabulary}{1.0\textwidth}{|l|L|l|L|}
\hline
    \textbf{Endpoint} & 
    \textbf{Type} & 
    \textbf{Parameters} & 
    \textbf{Description} \\
\hline
    /api/log/stats & GET & & Statistics from the current tree \\ \hline
    /api/log/graphviz & GET & & Outputs the current tree in the dot format \\ \hline
    /api/log/root & GET & & Gets the latest signed tree root \\ \hline
    /api/log/tree/append & POST & JSON object & Appends a JSON object to the tree \\ \hline
    /api/log/tree/id/<id> & GET & & Gets the given Node object from the database with '``id''\\ \hline
    /api/log/tree/hash/<hash> & GET & & Gets the given Node object from the database with ``hash''\\ \hline
    /api/log/tree/leaf/<id> & GET & & Gets the given leaf from the database with matching ``leaf\_index''\\ \hline
    /api/log/tree/validate/id/<id> & GET & & Gets the audit proof for a leaf matching the given ``id''\\ \hline
    /api/log/tree/validate/hash/<hash> & GET & & Gets the audit proof for a leaf matching the given ``hash''\\ \hline
    /api/log/tree/inclusion & POST & InclusionQuery & Provides the inclusion proof for the given query \\ \hline
    /api/log/tree/consistency & POST & ConsistencyQuery & Provides the consistency proof for the given query\\ \hline
\end{tabulary}
\caption{Transparency log API}
\label{api:transparency_log}
\end{table}


\begin{table}[hbtp]
\footnotesize
\centering
\settowidth\tymin{\textbf{Description}}
\setlength\extrarowheight{2pt}
\begin{tabulary}{1.0\textwidth}{|l|L|L|L|}
\hline
    \textbf{Endpoint} & 
    \textbf{Type} & 
    \textbf{Parameters} & 
    \textbf{Description} \\
\hline
    /api/rebuilder/submit & POST$^1$ & metadata, buildinfo& Implements something heyho lets go \\  \hline
    /api/rebuilder/revoke& POST$^1$ & leaf hash, reason, signature& Revokes the leaf with a reason  \\  \hline
    /api/rebuilder/fetch/<name>/<version>& GET & & Gets the entries for the given package \\  \hline
\end{tabulary}
\footnotesize{$^1$ Behind authentication}\\
\caption{Overlay API}
\label{api:Overlay API}
\end{table}


\section{APT Transport}\label{sec:apt_transport}
\begin{figure}[H]
  \centering
  \begin{sequencediagram}
    \newthread{A}{HTTP client}{}
    \newinst[1]{B}{intoto}{}
    \newinst[2]{C}{APT Server}{}
    \begin{messcall}{A}{100 Capabilities}{C}
        \begin{messcall}{C}{601 Configuration}{A}\end{messcall}
        \begin{messcall}{C}{600 URI Acquire}{A}\end{messcall}
        \begin{messcall}{A}{200 URI Start}{C}\end{messcall}
            \begin{messcall}{C}{201 URI Done}{B}
                \begin{sdblock}{Rebuilder verify}{}
                    \begin{messcall}{B}{201 URI Done}{A}\end{messcall}
                \end{sdblock}
            \end{messcall}
    \end{messcall}
  \end{sequencediagram}
\caption{intoto sequence diagram}
\label{lst:intoto_sequence_diagram}
\end{figure}


\blankpage
\end{document}
