% !TEX encoding = UTF-8 Unicode
%!TEX root = ../Main/thesis.tex
% !TEX spellcheck = en-US
%%=========================================
\documentclass[../Main/thesis.tex]{subfiles}
\begin{document}
\chapter{Technologies}\label{ch:technologies}
In this section we will take a look at the technology chosen for this project.

\subsection*{Python}
Python is a general purpose programming language created by Gudio van Rossum in
1994~\cite{python}. It's dynamically typed language, with a terse syntax and a wide selection
of built-in libraries for developers. Dynamically typed languages lends itself
nicely for rapid prototyping of experimental projects. A language like Rust or
Go, which are statically typed gives the developers some more issues prototyping
data structures as all types needs to be consistent and decided up on early.
With Go this becomes a worse problem with the lack of generics.

Python also has a wide selection of well maintained and frequently used
libraries that we can utilize for our project. Python is commonly used for
back-end development and web services inn general and has good libraries for
this.

\subsection*{flask}
Flask is a web framework for Python. It was created in 2010 by Armin Ronacher,
and is one of the two most widely popular web frameworks in python. It enabled
developers to easily create REST APIs with good debugging capabilities~\cite{flask}.

\subsection*{jinja}
One of the added features of using flask is that we get access to the templating
framework jinja which allows us to easily create webpages and interface them
with python values before being served to users. This allows us to easily create
webpages with the data used by the application~\cite{jinja}.

\subsection*{postgresql}
postgresql is a open-source relational database. It supports a wide number of
abstract datatypes, such as native support for JSON, along with good support for
concurrent operation. This enables easier development, along higher workloads
and scalability~\cite{pgsql}.

\subsection*{sqlalchemy}
sqlalchemy is a widely used object relation mapping library for python. It
support a wide selection of database backends and translates the raw database
data into usable python objects for easier interopability. This helps us to save
time by not having to map any values to our own data structures as everything
are written as native python classes~\cite{sqlalchemy}.

\begin{listing}[ht]
\caption{Example sqlalchemy model}
\label{lst:sqlalchemy}
\begin{minted}{python}
class User(db.Model):
    __tablename__ = "user"
    id = db.Column(db.Integer(), primary_key=True, autoincrement=True)
    created = db.Column(db.DateTime, default=datetime.utcnow)
    name = db.Column(db.String(96), index=True)

    def __repr__(self):
        return "<User: {}>".format(self.name)
\end{minted}
\end{listing}


\subsection*{Panda and matplotlib}
For graphing we will be using pandas and matplotlib. Both used a lot when it
comes to scientific computing with python. It enables powerful graphing
capabilities over simply dataformats, such as CSV, and are tightly integrated
into one another~\cite{pandas}\cite{matplotlib}.

\subsection*{Git and Github}%
\label{sub:git_and_github}
% TODO: Kilde på git
An important aspect of any project is to keep track of changes. For this project
the version control system ``git'' was used. It is a well known version control
system and widely used and deployed on a wide selection of providers. The
strongest point is being decentralized, where commits and code changes can be
done without internet access~\cite{git}. The code can also be pushed to multiple
repositories for backup purposes if any provider goes down during the
development of the project.

\subsection*{Deployment}%
\label{sub:deployment}
% TODO: Legge til source på uh-iaas
UH-IaaS is a cloud infrastructure provider created for research institution to
deploy software and build platform. They provide free resources for student
projects and the rebuilder infrastructure described in this thesis is currently
deployed on their platform. Extended testing of the implementations of this
thesis is also done on their platform.





\section*{Summary}\label{sec:summary-technologies} 
In this chapter we have taken a deeper look at the theory surrounding supply
chains, reproducible builds and merkle trees. In the next chapter we will be
taking a look at the current research being done towards shared attestation on
reproducible builds and package transparency logs.

\blankpage
\end{document}
