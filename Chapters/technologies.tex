% !TEX encoding = UTF-8 Unicode
%!TEX root = ../Main/thesis.tex
% !TEX spellcheck = en-US
%%=========================================
\documentclass[../Main/thesis.tex]{subfiles}
\begin{document}
\chapter{Technologies}\label{ch:technologies}
In this section we will take a look at the technology chosen for this project.

\subsection*{Python}
Python is a general purpose programming language created by Gudio van Rossum in
1994~\cite{python}. It's dynamically typed language, with a terse syntax and a wide selection
of built-in libraries for developers. Dynamically typed languages lends itself
nicely for rapid prototyping of experimental projects. A language like Rust or
Go, which are statically typed gives the developers some more issues prototyping
data structures as all types needs to be consistent and decided up on early.
With Go this becomes a worse problem with the lack of generics.

Python also has a wide selection of well maintained and frequently used
libraries that we can utilize for our project. Python is commonly used for
back-end development and web services inn general and has good libraries for
this.

\subsection*{Transparency Log}%
\label{sub:transparency_log_library}
The development of the transparency log requires some effort to properly
implement a Merkle tree with the appropriate proofs to so the implementation can
be useful. To do this we lifted some code from the Python library ``pymerkle''
by~\citeauthor{pymerkeltools} to aid in the development~\cite{pymerkeltools}.
Parts of the algorithms has been changed to accommodate the development goals of
this project. It was easier to lift code then find a suitable merkle tree
library to integrate with to provide the needed features.

\subsection*{flask}
Flask is a web framework for Python. It was created in 2010 by Armin Ronacher,
and is one of the two most widely popular web frameworks in Python. It enabled
developers to easily create REST APIs with good debugging
capabilities~\cite{flask}. This framework will be used to create the API
endpoints for the rebuilder.

\subsection*{jinja}
One of the added features of using flask is that we get access to the templating
framework jinja which allows us to easily create webpages and interface them
with Python values before being served to users. This allows us to easily create
webpages with the data used by the application~\cite{jinja}. We will be
utilizing this library to create the HTML webpages for the rebuilder.

\subsection*{postgresql}
postgresql is a open-source relational database. It supports a wide number of
abstract datatypes, such as native support for JSON, along with good support for
concurrent operation. This enables easier development, along higher workloads
and scalability~\cite{pgsql}. We decided to utilize PostgreSQL over other
technologies such as MySQL or sqlite because of the ability to embed JSON
structure directly into the code. 

\subsection*{sqlalchemy}
SQLAlchemy is a widely used object relation mapping library for Python. It
support a wide selection of database backends and translates the raw database
data into usable Python objects for easier interopability. This helps us to save
time by not having to map any values to our own data structures as everything
are written as native Python classes~\cite{sqlalchemy}.

\begin{listing}[ht]
\caption{Example SQLAlchemy model}
\label{lst:sqlalchemy}
\begin{minted}{python}
class User(db.Model):
    __tablename__ = "user"
    id = db.Column(db.Integer(), primary_key=True, autoincrement=True)
    created = db.Column(db.DateTime, default=datetime.utcnow)
    name = db.Column(db.String(96), index=True)

    def __repr__(self):
        return "<User: {}>".format(self.name)
\end{minted}
\end{listing}

Listing~\ref{lst:sqlalchemy} shows an example ORM model where we define a user
table. Each model has a created timestamp which defaults to the current time,
and a name value which defines a string for the name


\subsection*{Panda and matplotlib}
For graphing we will be using pandas and matplotlib. Both used a lot when it
comes to scientific computing with Python. It enables powerful graphing
capabilities over simply dataformats, such as CSV, and are tightly integrated
into one another~\cite{pandas}\cite{matplotlib}. We will be utilizing these
libraries to generate graphs during the evaluation phase on this thesis.

\subsection*{Git and Github}%
\label{sub:git_and_github}
An important aspect of any project is to keep track of changes. For this project
the version control system ``git'' was used. It is a well known version control
system and widely used and deployed on a wide selection of providers. The
strongest point is being decentralized, where commits and code changes can be
done without internet access~\cite{git}. The code can also be pushed to multiple
repositories for backup purposes if any provider goes down during the
development of the project.


\subsection*{Docker and containers}%
\label{sub:docker_and_containers}
One of the goals of this project is to make sure the testing is done in a
reproducible manner. To do this we need to make sure the environments are
consistent between testing, and that they can be reproducible rebuilt after the
testing is complete and into the future. To do this the development is done in
containers, namely a technology called Docker~\cite{docker}. Docker enables
reproducible containers with a given Linux distribution and dependencies. They
can be versioned and recreate environments across machines.

We will be utilizing this technology for extended testing for deployment and
data structure testing in this thesis. We will be providing scripts to recreate
the environments used to run the evaluations with docker and the container
technology.

\subsection*{curl}%
\label{sub:curl}
When developing an API is desirable to quickly test the functionality. For this
project we utilized the widely used and popular tool curl, which is written
by~\citeauthor{curl} and supports a wide array of URLs and data protocols for
data transfer~\cite{curl}. This allows us to quickly and effectively test and
display API end points we are going to be developing.


\subsubsection*{in-toto}%
\label{sub:in-toto}
in-toto is a framework to verify the integrity of a supply chain. It defines a
specification that details what steps should occur. As one supply chain could
define and utilize any number of steps its vital for this to be extensible, and
customize able. In-toto lets the specification detail who should perform the
step in the supply chain~\cite{in-toto}.

The layout describes what each step of the supply chain should contain. It can
be any expected commands, something the process should succeed running, any
expected material, things needed for the step in the chain to proceed, and any
products, artifacts created by the steps. These are described in a very small
language that contains keywords following regular expressions which should be
satisfied.

The link metadata is a JSON file that specifies what the values, and outputs of
the corresponding step should be. Evaluating the specification along with the
link metadata lets the users, or the organization, verify that the supply chain
has not been tampered with. 

\begin{listing}[H]
\begin{minted}[]{json}
{"signatures": [],
 "signed": {
  "_type": "layout",
  "expires": "2021-01-06T18:30:57Z",
  "inspect": [{
        "_type": "inspection",
        "expected_materials": [
             ["MATCH", "*.deb", "WITH", "PRODUCTS", "FROM", "rebuild"],
             ["DISALLOW", "*.deb"]],
        "expected_products": [],
        "name": "verify-reprobuilds",
        "run": ["/usr/bin/true"]}],
  "keys": {
        "2e7be98291270e3b7fca429a2210e99cff22017e":{
             "hashes": ["pgp+SHA2"],
             "keyid": "2e7be98291270e3b7fca429a2210e99cff22017e",
             "keyval": {"private": "", "public": {"e": "010001", "n": "e0da84bec..."}},
             "method": "pgp+rsa-pkcsv1.5",
             "type": "rsa"}},
  "steps": [{
        "_type": "step",
        "expected_products": [
             ["CREATE", "*.deb"],
             ["DISALLOW", "*.deb"]],
        "name": "rebuild",
        "pubkeys": ["2e7be98291270e3b7fca429a2210e99cff22017e"],
        "threshold": 1}]}}
\end{minted}
\caption{Example in-toto schema}
\label{lst:in-toto-schema}
\end{listing}

\begin{listing}[H]
\begin{minted}[]{json}
{"signatures": [
  {"keyid": "918b19596...",
   "other_headers": "0400010800...",
   "signature": "bc1d9776bf..."}],
 "signed": {
  "_type": "link",
  "name": "rebuild",
  "products": {
   "python-sshpubkeys_3.1.0-1_all.deb": {"sha256": "8e69d5cbdc..."},
   "python3-sshpubkeys_3.1.0-1_all.deb": {"sha256": "8234484139..."}}
}}
\end{minted}
\caption{Example linkmetadata file}
\label{lst:linkmetadata}
\end{listing}

Listing~\ref{lst:in-toto-schema} shows a in-toto schema. This is utilized to
show the needed steps in the supply chain. It aids in verifying who can sign off
on each step, what keys are used and what the acceptable output can be. To
verify this we need a link metadata which is an output of the supply chain.
Listing~\ref{lst:linkmetadata} is an example metadata file that contains a
signed layout with the products and checksums from the product. These two files
together allows us to verify that each step of a given supply chain has been
don.e

In this project we will be utilizing in-toto for the build attestation
verification part in the APT transport. In the in-toto APT transport we provide a
in-toto schema where the rebuilder provides the appropriate linkmetadata. Each
package downloaded by APT gets verified by running the in-toto process over
these two files.


\subsection*{Deployment}%
\label{sub:deployment}
UH-IaaS is a cloud infrastructure provider created for research institution to
deploy software and build platform~\cite{uhiaas}. They provide free resources for student
projects and the rebuilder infrastructure described in this thesis is currently
deployed on their platform. Extended testing of the implementations of this
thesis is also done on their platform.


\section*{Summary}\label{sec:summary-technologies} 
In this chapter we have taken a deeper look at the theory surrounding supply
chains, reproducible builds and Merkle trees. In the next chapter we will be
taking a look at the current research being done towards shared attestation on
reproducible builds and package transparency logs.
\end{document}
