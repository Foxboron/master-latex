% !TEX encoding = UTF-8 Unicode
%!TEX root = ../Main/thesis.tex
% !TEX spellcheck = en-US
%%=========================================
\documentclass[../Main/thesis.tex]{subfiles}
\begin{document}
\chapter{Technologies}\label{ch:technologies}
In this section we will take a look at the language and libraries used to
develop this project.

\subsection*{Python}
Python is a general purpose programming language created by Gudio van Rossum in
1994. It's dynamically typed language, with a terse syntax and a wide selection
of built-in libraries for developers. It's well suited for prototyping
technologies and has several well developed libraries.

\subsection*{flask}
Flask is a web framework for Python. It was created in 2010 by Armin Ronacher,
and is one of the two most widely popular web frameworks in python. It enabled
developers to easily create REST APIs with good debugging capabilities.

\subsection*{postgresql}
postgresql is a open-source relational database. It supports a wide number of
abstract datatypes, such as native support for JSON, along with good support for
concurrent operation. This enables easier development, along higher workloads
and scalability.

\subsection*{sqlalchemy}
sqlalchemy is a widely used object relation mapping library for python. It
support a wide selection of database backends and translates the raw database
data into usable python objects for easier interopability.

\subsection*{Panda and matplotlib}
For graphing we will be using pandas and matplotlib. Both used a lot when it
comes to scientific computing with python. It enables powerful graphing
capabilities over simply dataformats, such as CSV, and are tightly integrated
into one another.

\section*{Summary}\label{sec:summary-technologies} 
In this chapter we have taken a deeper look at the theory surrounding supply
chains, reproducible builds and merkle trees. In the next chapter we will be
taking a look at the current research being done towards shared attestation on
reproducible builds and package transparency logs.

\blankpage
\end{document}
