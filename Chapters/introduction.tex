% !TEX encoding = UTF-8 Unicode
% !TEX root = ../Main/thesis.tex
% !TEX spellcheck = en-US
%%=========================================
\documentclass[../Main/thesis.tex]{subfiles}
\begin{document}
\chapter{Introduction}\label{ch:introduction}
Distributing software is hard. For years Linux distributions have distributed
software by compiling them on build servers, or personal computers, and
submitting the packages to a central repository. The packages is then
distributed to a number of mirrors where users can get the latest software
updates. This enables users to get pre-compiled binaries instead of building
them by hand locally. But how do we know that the downloaded software has not
been tampered with? % TODO: Drøft

Targeted attacks against software supply chain is an increasing threat.  The
security company Symantec reported in 2018 that there was one supply chain
attack reported every month in 2017~\cite{symantec-istr-2018}, and in their 2019
report noted that targeted attacks has gone up 78\% in 2019
~\cite{symantec-istr-2019}. There has also been a growing concern amongst
security professionals about the dangers of supply chain
attacks~\cite{crowdstrike-supply-chain-attacks}. Companies are also struggling
to protect their software deliveries~\cite{cd-pipelines-case-study}.

% TODO: Bindeledd mellom

There has been a number of explicit attacks against supply chains that surrounds
Linux Distributions. Inappropriate commit access was achieved on the  Gentoo
build server~\cite{gentoo-compromise-2018}, Linux~\cite{linux-compromise-2011},
and even managed to install backdoors in a network equipment
vendor~\cite{Checkoway:2016:SAJ:2976749.2978395}~\cite{juniper-backdoor-advisory},

There have been attempts at breaking package
managers~\cite{Cappos:2008:LMA:1455770.1455841}, and several other instances of
attacks against Linux
distributions~\cite{gentoo-compromise-2003}~\cite{fedora-compromise-2011}~\cite{fedora-compromise-2008}.

In this thesis we will look into how reproducible builds enable distributions
and software authors to provide bit-for-bit identical binaries. We will extend
this concept and provide servers that take these packages and rebuild them to
provide additional attestation to the user whether or not a package is
reproducible. The build attestations will be using build metadata as defined
from the reproducible builds project.

We introduce an append-only package transparency log, closely resembling
certificate transparency logs, where we commit build attestations from the
rebuilders. This lets users easily verify rebuild attestations, along with
enabling the possibility of discovery of attacks against the distribution
supply-chain. We will further enhance this log by implementing the possibility
of revoking previous build attestations to further help users when verifying
packages.

The verification step of downloading packages will be performed by the Debian
package manager APT. APT supports the ability to have multiple transports for
retrieving packages from the web. To make sure we are able to query the needed
services for the verification, we will be providing a new package transport that
checks against a transparency log to detect tampering.

\section{Motivation}\label{sec:motivation}
The motivation this research is to investigate if transparency logs whether and
how this can give the users any new security guarantees on top of the rebuilder
verification. There has been no real-world deployment of publicly accessible
rebuilder infrastructure, and the integration of transparency logs into user
tools has not been investigated 

It should be noted that this research is primarily focused on the technical
implementation of such a system.

\section{Collaboration with New York University}\label{sec:collab}
This project has been done in collaboration with Lukas Puhringer and Santiago
Torres-Aires from the Secure Systems Lab department at New York University. This
thesis represents the individual research I have done. Together we have built a
complete rebuilder system for Debian packages and infrastructure. 

\begin{figure}[H]
\centering
\includegraphics[width=0.5\textwidth]{../Diagrams/architecture.pdf}
\caption{Rebuilder architecture overview}
\label{fig:rebuilder_architecture}
\end{figure}

The development describes in this thesis is the rewrite of the visualizer
component which provides an API to the user clients. The rewrite adds the
package transparency log capability of the system along with providing a more
refined API for the apt transport used to verify packages.

\section{Research questions and contributions}\label{sec:rq}
In this section we will present the research question of this thesis.  Along
with these research questions there is a motivation to contribute reproducible
research. All of the evaluations in this thesis have open-source code attached
to them.

\subsection*{RQ1}%
\label{sub:rq1}
\textit{Can a transparency log provide additional security guarantees, and if so, how?}
\\\\
The goal of the project is to see if we are capable of enhancing the visualizer
component of the rebuilder with more security guarantees. Transparency logs can
provide security gurantees if implemented correctly, namely the evidence that
the provided logs have or have not been tampered with. The current rebuilder
has no such feature and any build submissions can be tampered with after
publication.

\subsection*{RQ2}%
\label{sub:rq2}
\textit{Are we able to implement this into the current rebuilder verification process?}
\\\\
The current rebuilder verification process fetches plain text data from an
endpoint with no validation. To utilize the security features from a
transparency log, we would need to make sure they can be validated and
implemented in the APT package manager.

\subsection*{RQ3}%
\label{sub:rq3}
\textit{Can this be deployed in a real-world scenario?}
\\\\
Given a correct implementation of the transparency log, it would be interesting
to investigate how the resulting log implementation can work. Debian publishes
multiple packages each day, and we can see the amount of data the log would need
to consume and whether or not it is capable of consuming the data in a
real-world scenario.

\blankpage
\end{document}
