% !TEX encoding = UTF-8 Unicode
% !TEX root = ../Main/thesis.tex
% !TEX spellcheck = en-US
%%=========================================
\documentclass[../Main/thesis.tex]{subfiles}
\begin{document}
\chapter{Introduction}\label{ch:introduction}
Distributing software is hard. For years Linux distributions has distributed
software by compiling them on build servers, or personal computers, and
submitting them to a central repository. This is distributed to a number of
mirrors where users can get the latest software updates. This enables users to
get pre-compiled binaries instead of building it by hand locally. But how do we
know that the downloaded software has not been tampered with?

Packages are accompanied by the signature of the distribution when the users
fetches update. This enables the user to detect tampering during transit to the
user, but this gives the user no insight whether or not the package has been
tampered with during the supply-chain of the distribution.

In this thesis we will look into how reproducible builds enables distributions
and software authors to provide bit-for-bit identical binaries. We will extend
this concept and provide servers that takes these packages, and rebuilds them to
provide additional attestation to the user whether or not a package is
reproducible. The build attestations will be using build metadata as defined
from the reproducible builds project.

We introduce an append-only package transparency log, closely resembling
certificate transparency logs, where we commit build attestations from the
rebuilders. This lets users easily verify rebuild attestations, along with
enabling the possible discovery of attacks against the distribution
supply-chain.

The verification step of downloading packages will be performed by the Debian
package manager, apt, by providing a new package transport that checks against a
transparency log.


\section{Motivation}\label{sec:motivation}
The motivation this research is to see if transparency logs gives the users any
new security guarantees on top of the rebuilder verification. There has been no
real-world deployment of publicly accessible rebuilder infrastructure, and the
integration of transparency logs into user tools has not been investigated 

\section{Collaboration with New York University}\label{sec:collab}
This project has been done in collaboration with Lukas Puhringer and Santiago
Torres-Aires from the Secure Systems Lab department at New York University. This
thesis represents the individual research I have done. Together we have built a
complete rebuilder system for Debian packages and infrastructure. 

\begin{figure}[H]
\centering
\includegraphics[width=0.5\textwidth]{../Diagrams/architecture.pdf}
\caption{Rebuilder architecture overview}
\label{fig:rebuilder_architecture}
\end{figure}

The development describes in this thesis is the rewrite of the visualizer
component which provides an API to the user clients. The rewrite adds the
package transparency log capability of the system along with provided a more
refined API for the apt transport used to verify packages.

\section{Research questions}\label{sec:rq}
\begin{itemize}
    \item Does a package transparency log provide any additional security guarantees?
    \item Can this be deployed
    \item RQ3
\end{itemize}

\section{Organization of the thesis}\label{sec:organization}

\blankpage
\end{document}
