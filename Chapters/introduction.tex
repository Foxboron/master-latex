% !TEX encoding = UTF-8 Unicode
% !TEX root = ../Main/thesis.tex
% !TEX spellcheck = en-US
%%=========================================
\documentclass[../Main/thesis.tex]{subfiles}
\begin{document}
\chapter{Introduction}\label{ch:introduction}
Distributing software is a difficult task. For years Linux distributions have
distributed software by compiling them on build servers, and submitting the
packages to a central repository. The packages is then distributed to a number
of mirrors where users can get the latest software updates. This enables users
to get pre-compiled binaries instead of building them by hand locally. This is
an what is commonly referred to as the software supply chain. All the bits and
pieces that is involved from writing the code until delivery at the end user.

Targeted attacks against software supply chains is an increasing threat. The
security company Symantec reported in 2018 that there was one supply chain
attack reported every month in 2017~\cite{symantec-istr-2018}, and in their 2019
report noted that targeted attacks has gone up 78\% in 2019
~\cite{symantec-istr-2019}. There has also been a growing concern amongst
security professionals about the dangers of supply chain
attacks~\cite{crowdstrike-supply-chain-attacks} as companies are also struggling
to protect their software deliveries~\cite{cd-pipelines-case-study}. One example
is Juniper, a widely used vendor for network equipment, that realized a backdoor
was installed on equipment they distributed to
customers~\cite{Checkoway:2016:SAJ:2976749.2978395}~\cite{juniper-backdoor-advisory}.

There has also been a number of explicit attacks against supply chains that
surrounds Linux distributions. Inappropriate commit access was achieved on the
build server used by the Linux distribution Gentoo
~\cite{gentoo-compromise-2018}, and the server that holds the repository for the
Linux source code~\cite{linux-compromise-2011}.  In 2011 the Linux distribution
Fedora had one of their contributors targeted in a malicious
attack~\cite{fedora-compromise-2011}, coupled with an incident in 2008 where the
build infrastructure was compromised~\cite{fedora-compromise-2008}. There have
also been attempts at breaking package
managers~\cite{Cappos:2008:LMA:1455770.1455841}.

Packages used in programming languages have in recent years been high profile
targets for supply chain attacks. A popular package from the Ruby programming
language used by web developers was compromised in 2019. It started to include a
snippet of code allowing remote code execution on the developers
machine~\cite{malicious-ruby}. What is interesting about this case is that the
credentials of the authors where compromised, and the malicious code was never
found in the code repository. It was only available from the downloaded version
installed by the Ruby package manager.

Another supply chain attack was in the Javascript ecosystem where a widely used
dependency was targeted. The main developer had stopped development, and handed
the credentials to another developer who wanted to maintain this library over
email. This developer created a new version on Github, but the one uploaded to
the package registry used by the javascript language included a code snippet
that would attempt to steal crypt currency wallets on the
host~\cite{malicious-npm}. The library itself had been downloaded 800,000 times,
and is a widely used dependency.

Given the wide reach of supply chain attacks, we need to spend more effort
looking at how we can mitigate these risks for users and companies.

In this thesis we will look into how Reproducible Builds enable distributions
and software authors to provide bit-for-bit identical binaries. Reproducible
Builds is a set of software development practices and guidelines that enables
the creation of reproducible and deterministic compilation of software. This has
been a large focus by Linux distributions the past few years, and multiple
projects has joined the effort to make sure packages and software is
reproducible.

We have contributed with Secure Systems Lab at New York University, Reproducible
Builds and Debian to provide a system that rebuilds packages to provide
additional attestation to the user whether or not a package is reproducible by
integrating it into the Debian package manager APT.

The thesis takes this rebuilder system and introduce an append-only package
transparency log, closely resembling certificate transparency logs, where we
commit build attestations from the rebuilders. This lets users verify that
rebuild attestations have not been tampered with after publication. We will
further enhance this log by implementing the possibility of revoking previous
build attestations to further help users when verifying packages.

The verification step of downloading packages will be performed by the Debian
package manager APT. APT supports the ability to have multiple transports for
retrieving packages from the web. To make sure we are able to query the needed
services for the verification, we will be providing a new package transport that
checks against a transparency log to detect tampering.

\section{Motivation}\label{sec:motivation}
The motivation this research is to investigate if transparency logs whether and
how this can give the users any new security guarantees on top of the rebuilder
verification. There has been no real-world deployment of publicly accessible
rebuilder infrastructure, and the integration of transparency logs into user
tools has not been investigated 

The Reproducible Builds effort and Debian has through the years showed keen
interest in providing a secure rebuilder infrastructure, and this project is
an important step towards this overall goal of providing this to the end users
of the distribution.

It should be noted that this research is primarily focused on the technical
implementation of such a system.

\section{Collaboration with New York University}\label{sec:collab}
This project has been done in collaboration with Lukas Puehringer and Santiago
Torres-Arias from the Secure Systems Lab department at New York University. This
thesis represents the individual research I have done. Together we have built a
complete rebuilder system for Debian packages and infrastructure. 

\begin{figure}[H]
\centering
\includegraphics[width=0.5\textwidth]{../Diagrams/architecture.pdf}
\caption{Rebuilder architecture overview}
\label{fig:rebuilder_architecture}
\end{figure}

The development describes in this thesis is the rewrite of the visualizer
component which provides an API to the user clients. The rewrite adds the
package transparency log capability of the system along with providing a more
refined API for the APT transport used to verify packages.

\section{Research questions and contributions}\label{sec:rq}
In this section we will present the research question of this thesis.  Along
with these research questions there is a motivation to contribute reproducible
research. All of the evaluations in this thesis have open-source code attached
to them.

\subsection*{RQ1}%
\label{sub:rq1}
\textit{Can a transparency log provide additional security guarantees, and if so, how?}
\\\\
The goal of the project is to see if we are capable of enhancing the visualizer
component of the rebuilder with more security guarantees. Transparency logs can
provide security gurantees if implemented correctly, namely the evidence that
the provided logs have or have not been tampered with. The current rebuilder
has no such feature and any build submissions can be tampered with after
publication.

\subsection*{RQ2}%
\label{sub:rq2}
\textit{Are we able to implement this into the current rebuilder verification process?}
\\\\
The current rebuilder verification process fetches plain text data from an
endpoint with no validation. To utilize the security features from a
transparency log, we would need to make sure they can be validated and
implemented in the APT package manager.

\subsection*{RQ3}%
\label{sub:rq3}
\textit{Can this be deployed in a real-world scenario?}
\\\\
Given a correct implementation of the transparency log, it would be interesting
to investigate how the resulting log implementation can work. Debian publishes
multiple packages each day, and we can see the amount of data the log would need
to consume and whether or not it is capable of consuming the data in a
real-world scenario.
\end{document}
