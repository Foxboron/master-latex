% !TEX encoding = UTF-8 Unicode
%!TEX root = ../Main/thesis.tex
% !TEX spellcheck = en-US
%%=========================================
\documentclass[../Main/thesis.tex]{subfiles}
\begin{document}
\chapter{Conclusion}%
\label{ch:conclusion}
This chapter presents and summarized the thesis. We will go through the research
contributions, the limitations of the work we have done and the possible
direction this project can take in the future.

\section{Summary}%
\label{sec:summary}
In this thesis we have used Design Science research to develop a replacement for
the API front end used by the rebuilder infrastructure, called visualizer. We
have replaced the old implementation with a new extended version with support
for additional security properties by implementing a transparency log on top of
the API. We are capable of adding build submissions from the rebuilder, which
consists of a BUILDINFO file and a in-toto link metadata file. We are capable of
verifying whether or not published submissions have been tampered with after the
fact and integrate with the Debian package manager.

The development has gone over four iterations by establishing a set of
goals and making sure the goals are met after each iteration. After the
iterations on the visualizer rewrite, we had an integration phase into the
Debian package manager APT, to make sure we are able to utilize with the new
API.

Before the first iteration of the project we established the requirements of the
old API front end of the system. There where several API endpoints that had to be
implemented to ensure backwards compatibility. This was the first iteration
which was centered around making sure the code base is extensible and that the
old API endpoints was correctly implemented.

The second iteration implemented the Merkle tree data structure in a PostgreSQL
data storage back end. The implementation encompasses a dedicated API to
investigate and debug the API, along with appending raw JSON data structures to
the tree it self. The proofs we need to prove whether or not the submission have
been tampered with was implemented in this iteration: audit proofs and
consistency proofs. The visualizer also got an endpoint to visualize the Merkle
tree as a graph using the Graphviz library. The end result of this iteration was
a working Merkle tree implementation that works as a transparency log and
provide tamper evident logging. We are also capable of appending new elements
without having to re-hash the interior nodes of the tree.

The third iteration centered around the cryptographic parts of the Merkle tree.
All Merkle tree roots needs to be signed by a key pair to prove the given tree
root comes from the given log server. This implemented the integration of a
simple yet powerful library to aid in creating keys, signatures and the needed
API endpoints of the crypto API. We also made sure all new tree roots are signed
when appended leafs are added to the tree.

The fourth iteration implemented the transparency log overlay. We are for all
intents and purposes implementing logging of events, and these carry some
meaning which needs to be conveyed to the client. We implemented two kinds of
overlay entries for package submission inclusion, and revocation. This allows
the client to distinguish the meaning of different logged events. We are capable
of having multiple package submissions on the log, in the form of inclusions,
and revoke them if they where published or built erroneously. By appending
BUILDINFO files, which describes the build environment of packages, and in-toto
metadata, which is used to verify the integrity of the supply chain process, we
are able to retrieve and verify packages on the client side.  The result of this
is an abstracted API for the client to utilize when verifying packages on the
client side.

The final development section deals with the APT integration. APT is the Debian
package manager and is used to download packages from remote repositories. By
utilizing the transport system, and extending the transport written for the
rebuilder system, we where able to extend and improve on the transport to
utilize the new API and fetch in-toto link metadata from this process. We
implemented the needed extensions to the transport so we could validate the
consistency of rebuilders, and utilize the audit proofs of the fetched entries.

The evaluation of the prototype was done through technical testing where we
appended entries to the log and tested the response time as the process went on.
What we discovered is that the performance is not good enough for real-world use
as the current data storage does not scale well enough. However we are capable
of validate the proofs in the data storage and thus able of provide additional
security guarantees to the APT integration.

The research has shown the transparency log does provide additional security
guarantees, and that we are capable of implementing this into the current
verification process. However the current implementation is not sufficient for
real-world usage.

\section{Future Work}%
\label{sec:future_work}
The future work concerns the technical implementation of this project.
Moving the back end implementation away from the homegrown SQL data storage this
thesis presents, into a general data storage such as Trillian from Google would
give valuable insight into the possible scalability of this technology. Other
investigations into improvements of the data storage would be valuable to
enhance and contribute to the software supply chain security of Debian packages.


\section{Conclusion}%
\label{sec:conclusion_conclusion}
This research has presented the theory, development and evaluation of a
transparency log for a package rebuilder system. The research concludes that
even if the current implementation is not sufficient for real-world usage, we
are capable of having additional security guarantees with this project. However,
improving the efficiency of this system needs to be explored further.
\end{document}
