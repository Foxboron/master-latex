% !TEX encoding = UTF-8 Unicode
%!TEX root = ../Main/thesis.tex
% !TEX spellcheck = en-US
%%=========================================
\documentclass[../Main/thesis.tex]{subfiles}
\begin{document}
\chapter{Research Overview}\label{ch:research_overview}
In this chapter we will take a look at the current research that has gone into
the topics discussed in this thesis. We will go through current efforts and
related fields where research has been done.

% \section{Reproducible builds}\label{sec:repro}
% Reproducible builds is the challenge of producing bit-for-bit identical binaries
% given the same source code. This is important for projects that provides
% pre-compiled version of projects where the source-code is available. The efforts
% gone into reproducible builds has largely been done by Free and Open-Source
% communities the past years as this aligns greatly with the spirit of the
% communities.

\section{Software Distribution Transparency and Auditability}\label{sec:benjamin}
The initial work on package transparency logs, as a form of binary transparency,
was done by \citeauthor{1711.07278v1}. It implements an append-only merkle tree to
keep track of released versions of software into a package repository, namely
debian.\cite{1711.07278v1}

It details a novel hidden attack, where a backdoored package is distributed to
some users but not everyone. As the transparency log knows which package is the
correct one, it is possible to detect such attacks. The implementation also
defines log monitors, that peaks at inclusions and makes sure the logs are
operating properly and doesn't leave out information.


\section{CHAINIAC}\label{sec:chainiac}
CHAINIAC by \citeauthor*{kirill-niktin-2017} introduces a framework to help
collectively validate source-to-binary correspondence\cite{kirill-niktin-2017}.
This is done by introducing a cothority. Each developer commits their binaries
and corresponding source code to a merkle tree which is then signed by the
developers. This is built through a distributed system which will rebuild and
corroborate their results to aggregate signatures on packages. The number of
required signatures from this cothority is defined by the project and pushed to
a skipchain based update timeline.


\section{Contour}\label{sec:contour}
Contour by \citeauthor{1712.08427v2} is a system that implements binary
transparency by utilizing blockchains. In blockchains merkle trees are used to
store the data. In this implementation the resulting binaries are hashes on the
chain. The resulting merkle root is then hashed with previous roots to provide
the transaction and a block header. This is again distributed and secure from
split view attacks, where one server presents the client with a malicious view
of the world. Split view attacks enable the remote log to provide proofs to the
client which are not present else where. The solution to this is by adding a
consensus on top of the tree. The implementation was tested using the Python
packaging index, PyPi, and the debian package repository to test the solution.\cite{1712.08427v2}


\section{Go transparency log}\label{sec:go-transparency-log}
Go has been developing their dependency management the past year. With the
release of their new dependency manager, go mod, they now produce dependency
information along with a lockfile. The lockfile, go.sum, contains packages along
with versions and checksums.

A recent proposal by \citeauthor{russ-cos-and-filippo-valsorda} details how this
go.sum file would be commited to a transparency log to provide a verification
method for dependency releases in the go ecosystem. It implements a own database
to model merkle trees, and lets nodes validate and monitor the tree for
dependency inclusions. The idea is to check this log server when dependency
files are downloaded for proofs of releases.\cite{russ-cos-and-filippo-valsorda}

\section{TUF - The Update Framework}\label{sec:tuf}

\section*{Summary}\label{sec:researchoverview-summary}

\blankpage
\end{document}
